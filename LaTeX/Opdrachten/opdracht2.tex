\documentclass[a4paper,11pt]{article}
\usepackage[british]{babel}

\title{Assignment 2 - What is AI research?}
\author{Ysbrand Galama}
\date{\today}

\begin{document}
\pagenumbering{gobble}
\maketitle

\section*{Visual Recognition to Conserve Nature}
This project seems to be an empirical science. The goal of the project is to use existing techniques on a new problem. My approach would therefore be to collect data, research several papers which algorithm would fit best on the data, hypothesise the best one, test the algorithms on the data and see if the results are what was hypothesised.

The approach described above seems to follow the empirical cycle. Besides that, the other categories of {\it science} do not fit the description of the project. There will not be made a new model or algorithm, so it is not {\it design science}. There are no logic- or math principles involved to call it {\it formal science}. There could be argued that it is {\it science of the artificial}. As Maarten describes: {\it science of the artificial} makes and test an algorithm. However, the manner in which this project seems to be set up, {\it empirical science} fits is more.

\section*{Artificial Intelligence in Education}
The premeditation of this project is reasonable free. The student can bring his own idea how to solve it. However the main goal seems to be the designing of a new program/algorithm/model. Therefore I would categorise it as {\it design science}.

In this case there is also an argument to make that it is {\it science of the artificial}. The project could use a combination of several existing techniques and make a new program/algorithm/model to test and compare with other programs/algorithms/models. However, my approach would be to research what would be necessary to combine several programs/algorithms/models to make an automatic assessment system and then try to make this part. Therefore my approach would be as if it were {\it design science}.


\end{document}