\section{Conclusion}
\label{sec:Conclusion}
%Intepretatie uit resultaten
Several conclusions can be made from this project.
%Firstly, however, no conclusion can be made from the results of the Feed Forward Network.
%The implementation used in this project did not work, nevertheless usage of a neural network for training the second-to-last layer could work.
%In the discussion I will elaborate on this.
Firstly, the Support Vector Machine had promising results.
The tests of figure \ref{fig:c14} show that the linear model has one of the highest accuracies of the different hyperplanes, while also having the smallest time to train and test.
The RBF model shows less satisfying results.
It takes a large amount of time to train and test on the model, while not being able to gain a high accuracy.
The polynomial model trained on the train-set of 14k images shows an interesting trent: the higher the degree of the polynome, the lower the accuracy becomes.
In other words, for this problem -- detecting two classes from a large amount of feature-vectors -- the usage of a linear SVM works best.

The test of figure \ref{fig:lin1} shows an accuracy on plastic and animal detection of less than 1\permil.
However, because the model is trained on 70\% the images in the dataset, which consists mostly of consecutive frames of short film clips, there is a substantial amount of overfitting possible.

Therefore, as shown in figure \ref{fig:lin2} another test using the complete dataset from both above and below water viewpoints, shows how the linear SVM performs on different amounts of train data.
In this case overfitting when half the dataset is used is also probable, nevertheless, the model also shows accuracy of more than 90\% when trained on merely a small part of the train-set.
Therefore it can be concluded that this method, using the second-to-last layer of a pre-trained Convolutional Neural Network on an SVM, makes it possible to detect Plastic Soup in images.

Concluding from the localisation of plastic within the images is difficult.
The conducted test shows some results.
Several of the images show a higher confidance on locations where the consentration of plastic is higher.
However, this is not always the case; especialy with the location of marine-life.
More tests should be conducted with this technique, with possibly more annotated data, before a conclusion can be made.

Finally the usability of the promising method of using pre-trained CNNs to train on classes that were not in the original CNN is shown in this project.
Using a pre-trained CNN as a feature extractor and training another classifier is a simple technique to gain high accuracy while training on a small dataset.

%connections between existing theories and how the algorithm identifies plastic soup. including: what works? and what are the gaps?

%Figure \ref{fig:lin1} shows the improvement of accuracy when a larger train-set is used, while the enlarged train-set also increases the amount of time needed to train.
%In figures \ref{fig:c14} till \ref{fig:c14000} the increasing time needed for larger train-sets can also be seen between the figures.