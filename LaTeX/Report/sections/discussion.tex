\section{Discussion}
\label{sec:Discussion}
%intepretatie uit conclusie, beantwoord hypothese,
%\todo{kijken of subsections nodig zijn}
%====samenvatting
This project tried to contribute on solving the Plastic Soup problem.
The amount of floating plastic in the world's oceans could be very dangerous to the environment and society \citep{moore2011plastic}.
Automating the process could help with the clean-up and therefore a system that could detect plastic was constructed in this project.
For this purpose a dataset was build, on which a CNN was used for feature extraction.
On these features a SVM classifier was trained and tested against the labels of the dataset.

%==conclusie
%\paragraph{}
As shown in section \ref{sec:Results}, the best kernel for an SVM in this dataset is a linear hyperplane. This kernel had a high accuracy, even when being trained on small amounts of data, while not using much time to train or test.
The test of figure \ref{fig:lin1} shows an accuracy on plastic and animal detection with an error of less than 1\permil.
Therefore it can be concluded that this method, using the second-to-last layer of a pre-trained Convolutional Neural Network on an SVM, makes it possible to detect Plastic Soup in images.

A precise conclusion from the localisation experiment is difficult to draw.
The conducted test shows some results: several of the images show a higher confidence on locations where the concentration of plastic is higher.
However, this is not always the case; especially with the location of marine-life.
The current localisation pipeline does not work satisfactory, more tests should be conducted with this technique, with possibly more annotated data, to achieve better results.

%===interpretatie conclusie
%\paragraph{}
The accurate results of the SVM were be expected.
SVMs are known to work well in high dimensional data, as long as the amount of classes to be trained on is small.
The $4.096$ dimensional vector of the second-to-last layer of the CNN needed to be trained on two classes; each class had its own SVM trained on either showing the class or not.
However, because the model was trained on 70\% the images in the dataset, which consisted mostly of consecutive frames of short film clips, there is a substantial amount of overfitting possible.
\ifx\showmixi\undefined
Nevertheless, as the test of figure \ref{fig:lin3} shows, the pipeline shows a satisfactory accuracy when trained on the other viewpoint than tested and therefore refute a high accuracy solely based on over-fitting.
The low accuracy of the animal-class can be explained because the system is trained on fish and tested on birds, or vice versa.
Therefore a test is done on a mixed set of both viewpoints, which results are shown in figure \ref{fig:lin2}.
The usage of both the above water and below water viewpoints from the dataset in a single test, increases the amount of variance between the images.
The results show the same trend as before: gaining a high accuracy, even even trained on a small part of the dataset.
\else
Nevertheless, as the test of figure \ref{fig:lin2} shows, a high accuracy can also be obtained with a small amount of train-data, and a large test-set.
The test in this figure used both the above water and below water viewpoints from the dataset in a single test, increasing the amount of variance between the images.
\fi
In this case overfitting when half the dataset is used is very unlikely, since the model also shows an accuracy of more than 90\% when trained on merely a small part of the train-set.

The localisation experiments however had less satisfying results.
The confidence of detecting plastic or animals does not conform consequently to what is shown in the image.
This could be explained by the fact that the system learned to detect plastic from a distance.
While training for image classification, many pixels described the existence of plastic. In the localisation experiments however, a smaller amount of pixels was used for classification, making it harder for the pipeline to recognise the plastic on a different scale.
Furthermore, the sub-images are more zoomed than a full image.
Therefore the plastic has to be recognised in a more closed-up fashion than the system was trained on; instead of detecting much plastic together, single plastic objects now need to be recognized.
This could explain why the system had difficulties localising plastic and animals within the images.

%===evaluatie
%-weinig interessante dataset
%-CNN evalueren
%\todo{stukje over evaluatie onderzoek}
%\paragraph{}
In this project, several aspects could be addressed in order to improve the results.
Firstly the used dataset consisted of many similar frames, which increased the chance of overfitting to this particular dataset.
Furthermore, the classes that were trained on were not fairly distributed in the dataset.
An improved dataset containing more variance of images and fairer distributed classes, could be used to improve the results of this project.
This dataset should also contain labels on parts of the images, which could improve the training and testing of the localisation of plastic within images.
Another possible improvement is researching the usage of different CNNs.
A standard CNN was used to construct the feature-vector; no time was spent on researching if other networks would perform better.
Not only was assumed that the CNN worked properly and was therefore not further tested, also the parameters with which the SVM was tested were not statistical analysed; no cross-fold validation was conducted.
Nevertheless, as the accuracy of the pipeline was close to 100\%, no more time was taken to research these parameters than necessary.
The results show nonetheless how this pipeline could be used to detect Plastic Soup from image data.

%===vervolgonderzoek
%-localisatie verbeteren mbv BING oid
%\todo{stukje over vervolgonderzoek, oa verbeteren plastic-herkenning (ook locatie) en maken van autonomous agent voor schepen}
%\paragraph{}
This project showed that a pre-trained CNN in combination with a linear SVM can detect images containing either of two classes.
Further research in this domain could try to improve the system made in this project.
An improved dataset with more distributed classes and less similarity between the images in combination with training the data on more than one scale, could increase the performance on localisation of plastic within an image.
Projects using this method, should also keep in mind that cross-validation on the hyper-parameters could improve the results if they are not satisfactory.

An other example to improve the localisation of the plastic within the images, is the use of other algorithms.
If more complex algorithms are used for object detection in the image, the blobs resulting from these algorithms could be input for the pipeline, instead of the now used segmentation-pyramid.
One of the algorithms that could be used for this is BING \citep{BingObj2014}, however this project did not have the time to implement one of these algorithms.

Before the system proposed in this project could be used for real-world applications, more research is needed.
However, this project showed the possibility of using state-of-the-art Computer Vision techniques to detect Plastic Soup.
More research building on the results of this project could result in applications that can detect Plastic Soup.

%===afsluiting
%\paragraph{}
This project is one of the many recent projects that show the possibilities of Convolutional Neural Networks.
The usability of the promising method of using pre-trained CNNs to train on classes that were not in the original CNN is shown in this project.
Using a pre-trained CNN as a feature extractor and training an SVM is a simple technique to gain high accuracy while training on a small dataset.
Besides that, it also shows how Artificial Intelligence can be used to help solving environmental problems.
I do not think the hippie movement in the '60s could imagine that AI could solve their problems for them today.

%\subsection{Future research}
%Evatuatie onderzoek, vervolgonderzoek



%%==OLD
%This project tried to begin the build of autonomous agents that could clean up the ocean of plastic.
%An agent was not build, however a sizeable start was made with the vision of such an agent.
%Several state-of-the-art algorithms were used on a dataset of images containing floating plastic, animals, both or none to classify the images in one of the classes.
%One of the algorithms -- the Feed Forward Network -- trained on the second-to-last layer of a Convolutional Neural Network, did not work satisfactory.

%There are several possible reasons the Feed Forward Network used in this project did not perform well.
%Because the implementation was not part of a \texttt{Python} library, there could be bugs.
%This however does not seem probable, because the implementation was tested with a xor dataset which performed properly.

%More likely is the normalisation of the data.
%The second-to-last layer of the CNN was not normalised, while the FNN implementation used ${tanh}()$ to approximate the Sigmoid function.
%However, even a normalisation of the data did not affect the outcome.
%Bottom-line, the network was not able to learn the classes from the data.