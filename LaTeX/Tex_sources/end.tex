\vfill
\bibliographystyle{abbrvnat}
\bibliography{Tex_sources/bib}

\clearpage

\begin{appendix}
\renewcommand{\thesubsection}{\thesection.\roman{subsection}}
\renewcommand{\thesubsubsection}{\thesubsection - \arabic{subsubsection}}
\addtocontents{toc}{\setcounter{tocdepth}{2}}

\section{\texttt{Python} code of the project}
\label{sec:ap-code}

\section{Output of \texttt{Python} code}
\label{sec:ap-out}

\subsection{SVM}
\ifx\showapp\undefined
\begin{tabular}{ p{0.2\textwidth } | p{0.8\textwidth } }
abbrivation & meaning \\ \hline
type & which type of model used for the fitting hyperplane; poly mean polynomial, where the last digit stands for the degree; rbf is gaussian, where the last digit stands for gamma \\
Ttrain & the amount of time in seconds the complete training took \\
Ttest & the amount of time in seconds the complete testing took \\
Dsize & the amount of data-points used for training the model \\
Vsize & the amount of data-points used for testing the model \\
Bco & the amount of test-points that were correct\footnote{correct means here having the same output as the label, so both \texttt{11} and \texttt{00} are considered correct i.e. an iff} on both classes \\
Pco & the amount of test-points that were correct only on the plastic class \\
Aco & the amount of test-points that were correct only on the animal class \\
Bac & the accuracy on both classes i.e. $\frac{Bco}{Vsize}$ \\
Pac & the accuracy on the plastic class i.e. $\frac{Bco+Pco}{Vsize}$ \\
Aac & the accuracy on the animals class i.e. $\frac{Bco+Aco}{Vsize}$
\end{tabular}
{\small
\begin{longtable}{r|r|r|r|r|r|r|r|r|r|r|r}
\subsubsection{Train and validate below water}
\begin{longtable}{r|r|r|r|r|r|r|r|r|r|r|r}
          type &  Ttrain &   Ttest & Dsize & Vsize &   Bco &   Pco &   Aco &   Bfa &   Bac &   Pac &   Aac \\
   poly,1.0,2  &     0.0 &     0.7 &    14 &  2061 &  1144 &   378 &   453 &    86 & 0.555 & 0.738 & 0.775 \\
   poly,1.0,3  &     0.0 &     1.0 &    14 &  2061 &  1099 &   410 &   497 &    55 & 0.533 & 0.732 & 0.774 \\
   poly,1.0,4  &     0.0 &     1.0 &    14 &  2061 &   890 &   593 &   536 &    42 & 0.432 & 0.720 & 0.692 \\
   poly,1.0,5  &     0.0 &     0.8 &    14 &  2061 &   779 &   698 &   548 &    36 & 0.378 & 0.717 & 0.644 \\
   poly,1.0,6  &     0.0 &     1.2 &    14 &  2061 &   988 &   489 &   539 &    45 & 0.479 & 0.717 & 0.741 \\
   poly,1.0,7  &     0.0 &     0.9 &    14 &  2061 &  1237 &   235 &   352 &   237 & 0.600 & 0.714 & 0.771 \\
   poly,1.0,8  &     0.0 &     1.2 &    14 &  2061 &  1216 &   252 &   204 &   389 & 0.590 & 0.712 & 0.689 \\
   poly,1.0,9  &     0.0 &     1.1 &    14 &  2061 &  1214 &   251 &   108 &   488 & 0.589 & 0.711 & 0.641 \\
  rbf,1.0,0.0  &     0.0 &     0.9 &    14 &  2061 &  1174 &   271 &    47 &   569 & 0.570 & 0.701 & 0.592 \\
  rbf,1.0,0.1  &     0.0 &     0.8 &    14 &  2061 &  1174 &   271 &    47 &   569 & 0.570 & 0.701 & 0.592 \\
  rbf,1.0,0.2  &     0.0 &     1.2 &    14 &  2061 &  1174 &   271 &    47 &   569 & 0.570 & 0.701 & 0.592 \\
  rbf,1.0,0.3  &     0.0 &     1.2 &    14 &  2061 &   392 &  1053 &   594 &    22 & 0.190 & 0.701 & 0.478 \\
  rbf,1.0,0.4  &     0.0 &     1.0 &    14 &  2061 &   354 &  1091 &   594 &    22 & 0.172 & 0.701 & 0.460 \\
  rbf,1.0,0.5  &     0.0 &     1.2 &    14 &  2061 &   338 &  1107 &   594 &    22 & 0.164 & 0.701 & 0.452 \\
  rbf,1.0,0.6  &     0.0 &     0.8 &    14 &  2061 &   330 &  1115 &   594 &    22 & 0.160 & 0.701 & 0.448 \\
  rbf,1.0,0.7  &     0.0 &     0.8 &    14 &  2061 &   325 &  1120 &   594 &    22 & 0.158 & 0.701 & 0.446 \\
  rbf,1.0,0.8  &     0.0 &     1.2 &    14 &  2061 &   317 &  1128 &   594 &    22 & 0.154 & 0.701 & 0.442 \\
  rbf,1.0,0.9  &     0.0 &     0.7 &    14 &  2061 &   316 &  1129 &   594 &    22 & 0.153 & 0.701 & 0.442 \\
  rbf,1.0,1.0  &     0.0 &     0.7 &    14 &  2061 &   316 &  1129 &   594 &    22 & 0.153 & 0.701 & 0.442 \\
   linear,1.0  &     0.0 &     0.6 &    14 &  2061 &  1177 &   350 &   401 &   133 & 0.571 & 0.741 & 0.766 \\
          type &  Ttrain &   Ttest & Dsize & Vsize &   Bco &   Pco &   Aco &   Bfa &   Bac &   Pac &   Aac \\
   poly,1.0,2  &     0.4 &     4.5 &   140 &  2061 &  1890 &    71 &    45 &    55 & 0.917 & 0.951 & 0.939 \\
   poly,1.0,3  &     0.5 &     5.6 &   140 &  2061 &  1876 &    77 &    50 &    58 & 0.910 & 0.948 & 0.934 \\
   poly,1.0,4  &     0.4 &     6.3 &   140 &  2061 &  1847 &    91 &    56 &    67 & 0.896 & 0.940 & 0.923 \\
   poly,1.0,5  &     0.4 &     7.1 &   140 &  2061 &  1820 &   105 &    68 &    68 & 0.883 & 0.934 & 0.916 \\
   poly,1.0,6  &     0.6 &     7.5 &   140 &  2061 &  1789 &   114 &    76 &    82 & 0.868 & 0.923 & 0.905 \\
   poly,1.0,7  &     0.6 &     7.7 &   140 &  2061 &  1759 &   123 &    85 &    94 & 0.853 & 0.913 & 0.895 \\
   poly,1.0,8  &     0.4 &     7.6 &   140 &  2061 &  1733 &   133 &    78 &   117 & 0.841 & 0.905 & 0.879 \\
   poly,1.0,9  &     0.6 &     8.3 &   140 &  2061 &  1713 &   143 &    75 &   130 & 0.831 & 0.901 & 0.868 \\
  rbf,1.0,0.0  &     0.4 &     7.2 &   140 &  2061 &  1235 &   253 &    89 &   484 & 0.599 & 0.722 & 0.642 \\
  rbf,1.0,0.1  &     0.4 &     7.2 &   140 &  2061 &  1140 &   305 &    23 &   593 & 0.553 & 0.701 & 0.564 \\
  rbf,1.0,0.2  &     0.6 &    10.3 &   140 &  2061 &  1140 &   305 &    23 &   593 & 0.553 & 0.701 & 0.564 \\
  rbf,1.0,0.3  &     0.6 &    10.1 &   140 &  2061 &  1140 &   305 &    22 &   594 & 0.553 & 0.701 & 0.564 \\
  rbf,1.0,0.4  &     0.6 &     9.5 &   140 &  2061 &  1140 &   305 &    22 &   594 & 0.553 & 0.701 & 0.564 \\
  rbf,1.0,0.5  &     0.6 &    10.4 &   140 &  2061 &  1140 &   305 &    22 &   594 & 0.553 & 0.701 & 0.564 \\
  rbf,1.0,0.6  &     0.4 &     7.8 &   140 &  2061 &  1140 &   305 &    22 &   594 & 0.553 & 0.701 & 0.564 \\
  rbf,1.0,0.7  &     0.4 &     7.1 &   140 &  2061 &  1140 &   305 &    22 &   594 & 0.553 & 0.701 & 0.564 \\
  rbf,1.0,0.8  &     0.6 &    10.4 &   140 &  2061 &  1140 &   305 &    22 &   594 & 0.553 & 0.701 & 0.564 \\
  rbf,1.0,0.9  &     0.4 &     6.4 &   140 &  2061 &  1140 &   305 &    22 &   594 & 0.553 & 0.701 & 0.564 \\
  rbf,1.0,1.0  &     0.4 &     6.8 &   140 &  2061 &  1140 &   305 &    22 &   594 & 0.553 & 0.701 & 0.564 \\
   linear,1.0  &     0.2 &     3.1 &   140 &  2061 &  1851 &   101 &    54 &    55 & 0.898 & 0.947 & 0.924 \\
          type &  Ttrain &   Ttest & Dsize & Vsize &   Bco &   Pco &   Aco &   Bfa &   Bac &   Pac &   Aac \\
   poly,1.0,2  &    10.1 &    17.2 &  1400 &  2061 &  2048 &     8 &     5 &     0 & 0.994 & 0.998 & 0.996 \\
   poly,1.0,3  &    14.1 &    20.5 &  1400 &  2061 &  2049 &     7 &     5 &     0 & 0.994 & 0.998 & 0.997 \\
   poly,1.0,4  &    13.0 &    17.1 &  1400 &  2061 &  2047 &     9 &     5 &     0 & 0.993 & 0.998 & 0.996 \\
   poly,1.0,5  &    12.7 &    18.8 &  1400 &  2061 &  2044 &    11 &     6 &     0 & 0.992 & 0.997 & 0.995 \\
   poly,1.0,6  &    20.6 &    30.3 &  1400 &  2061 &  2042 &    13 &     6 &     0 & 0.991 & 0.997 & 0.994 \\
   poly,1.0,7  &    22.8 &    32.9 &  1400 &  2061 &  2032 &    18 &    11 &     0 & 0.986 & 0.995 & 0.991 \\
   poly,1.0,8  &    16.1 &    23.3 &  1400 &  2061 &  2024 &    21 &    11 &     5 & 0.982 & 0.992 & 0.987 \\
   poly,1.0,9  &    23.2 &    38.5 &  1400 &  2061 &  2013 &    23 &    15 &    10 & 0.977 & 0.988 & 0.984 \\
  rbf,1.0,0.0  &    29.8 &    40.6 &  1400 &  2061 &  2040 &    10 &    11 &     0 & 0.990 & 0.995 & 0.995 \\
  rbf,1.0,0.1  &    64.7 &    93.2 &  1400 &  2061 &  1140 &   305 &    38 &   578 & 0.553 & 0.701 & 0.572 \\
  rbf,1.0,0.2  &    58.7 &    77.5 &  1400 &  2061 &  1140 &   305 &    24 &   592 & 0.553 & 0.701 & 0.565 \\
  rbf,1.0,0.3  &    57.6 &    79.9 &  1400 &  2061 &  1140 &   305 &    22 &   594 & 0.553 & 0.701 & 0.564 \\
  rbf,1.0,0.4  &    58.6 &    82.3 &  1400 &  2061 &  1140 &   305 &    22 &   594 & 0.553 & 0.701 & 0.564 \\
  rbf,1.0,0.5  &    58.2 &    82.1 &  1400 &  2061 &  1140 &   305 &    22 &   594 & 0.553 & 0.701 & 0.564 \\
  rbf,1.0,0.6  &    51.7 &    91.9 &  1400 &  2061 &  1140 &   305 &    22 &   594 & 0.553 & 0.701 & 0.564 \\
  rbf,1.0,0.7  &    50.9 &    83.2 &  1400 &  2061 &  1140 &   305 &    22 &   594 & 0.553 & 0.701 & 0.564 \\
  rbf,1.0,0.8  &    48.9 &    87.9 &  1400 &  2061 &  1140 &   305 &    22 &   594 & 0.553 & 0.701 & 0.564 \\
  rbf,1.0,0.9  &    54.5 &    45.1 &  1400 &  2061 &  1140 &   305 &    22 &   594 & 0.553 & 0.701 & 0.564 \\
  rbf,1.0,1.0  &    54.2 &    78.5 &  1400 &  2061 &  1140 &   305 &    22 &   594 & 0.553 & 0.701 & 0.564 \\
   linear,1.0  &     5.6 &     8.3 &  1400 &  2061 &  2047 &     6 &     8 &     0 & 0.993 & 0.996 & 0.997 \\
          type &  Ttrain &   Ttest & Dsize & Vsize &   Bco &   Pco &   Aco &   Bfa &   Bac &   Pac &   Aac \\
   poly,1.0,2  &  1124.4 &   155.7 & 14000 &  2061 &  2015 &    24 &    22 &     0 & 0.978 & 0.989 & 0.988 \\
   poly,1.0,3  &  2068.0 &   277.0 & 14000 &  2061 &  1987 &    47 &    23 &     4 & 0.964 & 0.987 & 0.975 \\
   poly,1.0,4  &  3331.6 &   459.8 & 14000 &  2061 &  1777 &   162 &    84 &    38 & 0.862 & 0.941 & 0.903 \\
   poly,1.0,5  &  4678.5 &   646.8 & 14000 &  2061 &  1527 &   231 &    87 &   216 & 0.741 & 0.853 & 0.783 \\
   poly,1.0,6  &  5198.9 &   696.6 & 14000 &  2061 &  1249 &   278 &    55 &   479 & 0.606 & 0.741 & 0.633 \\
   poly,1.0,7  &  5282.2 &   706.6 & 14000 &  2061 &  1192 &   305 &    22 &   542 & 0.578 & 0.726 & 0.589 \\
   poly,1.0,8  &  5193.3 &   719.4 & 14000 &  2061 &  1173 &   305 &    24 &   559 & 0.569 & 0.717 & 0.581 \\
   poly,1.0,9  &  5083.9 &   714.2 & 14000 &  2061 &  1157 &   305 &    22 &   577 & 0.561 & 0.709 & 0.572 \\
  rbf,1.0,0.0  &   586.3 &    86.2 & 14000 &  2061 &  2043 &     6 &    12 &     0 & 0.991 & 0.994 & 0.997 \\
  rbf,1.0,0.1  & 24560.3 &  1067.4 & 14000 &  2061 &  1230 &   285 &   130 &   416 & 0.597 & 0.735 & 0.660 \\
  rbf,1.0,0.2  & 25139.6 &  1069.5 & 14000 &  2061 &  1152 &   304 &    57 &   548 & 0.559 & 0.706 & 0.587 \\
  rbf,1.0,0.3  & 25104.7 &  1059.1 & 14000 &  2061 &  1142 &   305 &    31 &   583 & 0.554 & 0.702 & 0.569 \\
  rbf,1.0,0.4  & 25389.9 &  1052.6 & 14000 &  2061 &  1142 &   305 &    26 &   588 & 0.554 & 0.702 & 0.567 \\
  rbf,1.0,0.5  & 24877.4 &  1055.4 & 14000 &  2061 &  1142 &   305 &    21 &   593 & 0.554 & 0.702 & 0.564 \\
  rbf,1.0,0.6  & 26216.1 &  1062.9 & 14000 &  2061 &  1142 &   305 &    20 &   594 & 0.554 & 0.702 & 0.564 \\
  rbf,1.0,0.7  & 25266.2 &   618.3 & 14000 &  2061 &  1142 &   305 &    20 &   594 & 0.554 & 0.702 & 0.564 \\
  rbf,1.0,0.8  & 26166.7 &  1072.2 & 14000 &  2061 &  1141 &   305 &    21 &   594 & 0.554 & 0.702 & 0.564 \\
  rbf,1.0,0.9  & 18654.6 &   607.0 & 14000 &  2061 &  1141 &   305 &    21 &   594 & 0.554 & 0.702 & 0.564 \\
  rbf,1.0,1.0  & 16250.9 &   399.1 & 14000 &  2061 &  1140 &   305 &    22 &   594 & 0.553 & 0.701 & 0.564 \\
   linear,1.0  &    74.6 &     8.4 & 14000 &  2061 &  2058 &     2 &     1 &     0 & 0.999 & 1.000 & 0.999 \\
\end{longtable}
\subsubsection{Train and test below water}
\begin{longtable}{r|r|r|r|r|r|r|r|r|r|r|r}
    type &  Ttrain &   Ttest & Dsize & Vsize &   Bco &   Pco &   Aco &   Bfa &   Bac &   Pac &   Aac \\
    linear,1.0 &     0.0 &     5.9 &     9 &  4123 &  2228 &   797 &   748 &   350 & 0.540 & 0.734 & 0.722 \\
    linear,1.0 &     0.0 &     7.8 &    18 &  4123 &  2402 &   726 &   795 &   200 & 0.583 & 0.759 & 0.775 \\
    linear,1.0 &     0.4 &    14.5 &    90 &  4123 &  3534 &   215 &   213 &   161 & 0.857 & 0.909 & 0.909 \\
    linear,1.0 &     0.8 &    19.6 &   180 &  4123 &  3753 &   136 &   148 &    86 & 0.910 & 0.943 & 0.946 \\
    linear,1.0 &    10.0 &    27.5 &   900 &  4123 &  4057 &    25 &    32 &     9 & 0.984 & 0.990 & 0.992 \\
    linear,1.0 &    21.1 &    29.1 &  1800 &  4123 &  4092 &    13 &    15 &     3 & 0.992 & 0.996 & 0.996 \\
    linear,1.0 &    82.5 &    23.2 &  9000 &  4123 &  4118 &     3 &     2 &     0 & 0.999 & 1.000 & 0.999 \\
    linear,1.0 &   123.8 &    16.8 & 18000 &  4123 &  4119 &     2 &     2 &     0 & 0.999 & 1.000 & 1.000 \\
\end{longtable}
\ifx\showmixi\undefined
\subsubsection{Train on below water, test on above water}
\begin{longtable}{r|r|r|r|r|r|r|r|r|r|r|r}
          type &  Ttrain &   Ttest & Dsize & Vsize &   Bco &   Pco &   Aco &   Bfa &   Bac &   Pac &   Aac \\
    linear,1.0 &     0.0 &     4.4 &     9 &  3311 &  1396 &  1530 &    97 &   288 & 0.422 & 0.884 & 0.451 \\
    linear,1.0 &     0.0 &     6.8 &    18 &  3311 &  1578 &  1348 &   173 &   212 & 0.477 & 0.884 & 0.529 \\
    linear,1.0 &     0.4 &    12.6 &    90 &  3311 &  1673 &  1208 &   167 &   263 & 0.505 & 0.870 & 0.556 \\
    linear,1.0 &     0.6 &    15.5 &   180 &  3311 &  1615 &  1090 &   339 &   267 & 0.488 & 0.817 & 0.590 \\
    linear,1.0 &     9.8 &    21.7 &   900 &  3311 &  1684 &   999 &   277 &   351 & 0.509 & 0.810 & 0.592 \\
    linear,1.0 &    20.7 &    23.1 &  1800 &  3311 &  1757 &  1042 &   242 &   270 & 0.531 & 0.845 & 0.604 \\
    linear,1.0 &    79.0 &    18.7 &  9000 &  3311 &  1659 &  1301 &   248 &   103 & 0.501 & 0.894 & 0.576 \\
    linear,1.0 &   118.4 &    13.5 & 18000 &  3311 &  1685 &  1272 &   237 &   117 & 0.509 & 0.893 & 0.580 
\end{longtable}
\subsubsection{Train on above water, test on below water}
\begin{longtable}{r|r|r|r|r|r|r|r|r|r|r|r}
          type &  Ttrain &   Ttest & Dsize & Vsize &   Bco &   Pco &   Aco &   Bfa &   Bac &   Pac &   Aac \\
    linear,1.0 &     0.0 &     5.9 &     9 &  4123 &  2147 &   771 &   282 &   923 & 0.521 & 0.708 & 0.589 \\
    linear,1.0 &     0.0 &     7.6 &    18 &  4123 &  2218 &   731 &   171 &  1003 & 0.538 & 0.715 & 0.579 \\
    linear,1.0 &     0.4 &    14.5 &    90 &  4123 &  1724 &   909 &   482 &  1008 & 0.418 & 0.639 & 0.535 \\
    linear,1.0 &     0.6 &    17.0 &   180 &  4123 &  1660 &  1073 &   573 &   817 & 0.403 & 0.663 & 0.542 \\
    linear,1.0 &     6.6 &    20.6 &   900 &  4123 &  1735 &   954 &   413 &  1021 & 0.421 & 0.652 & 0.521 \\
    linear,1.0 &    13.5 &    18.2 &  1800 &  4123 &  1894 &   872 &   378 &   979 & 0.459 & 0.671 & 0.551 \\
    linear,1.0 &    46.2 &    12.4 &  9000 &  4123 &  1885 &  1041 &   284 &   913 & 0.457 & 0.710 & 0.526 \\
    linear,1.0 &    55.9 &     9.9 & 18000 &  4123 &  1851 &   975 &   375 &   922 & 0.449 & 0.685 & 0.540 
\end{longtable}
\fi

\subsubsection{Train and test mixed dataset}
\begin{longtable}{r|r|r|r|r|r|r|r|r|r|r|r}
          type &  Ttrain &   Ttest & Dsize & Vsize &   Bco &   Pco &   Aco &   Bfa &   Bac &   Pac &   Aac \\
   linear,1.0  &     0.0 &    17.0 &     9 & 18583 & 11538 &  2972 &   846 &  3227 & 0.621 & 0.781 & 0.666 \\
   linear,1.0  &     0.0 &    33.1 &    18 & 18583 & 10853 &  3951 &  2316 &  1463 & 0.584 & 0.797 & 0.709 \\
   linear,1.0  &     0.3 &    77.8 &    90 & 18583 & 15465 &  1336 &  1308 &   474 & 0.832 & 0.904 & 0.903 \\
   linear,1.0  &     1.4 &    95.4 &   180 & 18583 & 16713 &   827 &   686 &   357 & 0.899 & 0.944 & 0.936 \\
   linear,1.0  &    12.1 &   169.0 &   900 & 18583 & 18134 &   243 &   196 &    10 & 0.976 & 0.989 & 0.986 \\
   linear,1.0  &    31.5 &   186.6 &  1800 & 18583 & 18325 &   139 &   112 &     7 & 0.986 & 0.994 & 0.992 \\
   linear,1.0  &   169.7 &   149.2 &  9000 & 18583 & 18544 &    24 &    14 &     1 & 0.998 & 0.999 & 0.999 \\
   linear,1.0  &   293.2 &   113.2 & 18000 & 18583 & 18568 &     7 &     7 &     1 & 0.999 & 1.000 & 1.000 \\
\end{longtable}
\end{longtable}
}
\fi
\end{appendix}
